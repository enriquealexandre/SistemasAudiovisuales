\documentclass[es,practica]{uah}

\usepackage{hyperref}

\tema{A3}
\titulo{Mezcla de Audio}{}

\begin{document}

\titulacion{Grados Informática}
\asignatura{Sistemas Audiovisuales y Aplicaciones Multimedia}{}
\curso{2021/2022}

\maketitle




\section{Descripción de la práctica}

El objetivo es crear una mezcla de una canción a partir de las pistas de sus instrumentos aislados. 

Podéis elegir la canción que queráis siempre que tengáis las pistas de los instrumentos por separado para poder trabajar. Os recomiendo echarle un ojo a la siguiente página: \href{https://cambridge-mt.com/ms/mtk/}{https://cambridge-mt.com/ms/mtk/} o utilizar cualquier otra que queráis. También podéis descargaros efectos de sonido, si os apetece, de freesound.org.

Para la mezcla debéis ajustar los siguientes elementos:

\begin{itemize}
\item Nivel relativo de las pistas.
\item Panorámica, para conseguir separar bien los instrumentos.
\item Uso de ecualizadores para mejorar el sonido y aumentar la separación.
\item Uso de distintos efectos en las pistas individuales para modificar su sonido, añadir pegada, etc.
\end{itemize}

\section{Material a entregar}

\begin{enumerate}
\item Adjunta un archivo mp3 con la mezcla realizada, así como un breve guión en formato texto con todos los pasos que has seguido para realizar dicha mezcla.
\end{enumerate}




\end{document}

	
