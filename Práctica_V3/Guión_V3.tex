\documentclass[es,practica]{uah}

\usepackage{hyperref}

\tema{V3}
\titulo{Edición de vídeo con DaVinci Resolve}{}

\begin{document}

\titulacion{Grados Informática}
\asignatura{Sistemas Audiovisuales y Aplicaciones Multimedia}{}
\curso{2021/2022} 

\maketitle



\section{Editor de vídeo DaVinci Resolve}


El objetivo de esta práctica es familiarizarse con las funciones básicas del programa de edición de vídeo DaVinci Resolve. En el vídeo de introducción que hay en el Aula Virtual podéis ver algunas de estas funciones así como el funcionamiento general del programa. Si queréis profundizar algo más, o si os interesa utilizar este programa para la práctica final, tenéis mucha información y tutoriales con materiales para practicar en la propia página de DaVinci\footnote{\url{https://www.blackmagicdesign.com/es/products/davinciresolve/training}}.

Para realizar esta práctica necesitaréis varios archivos cortos de vídeo, grabados por vosotros mismos o descargados de internet, al menos uno de ellos grabado con un croma. Podéis grabaros vosotros mismos con un croma simplemente utilizando un fondo con un color uniforme. Lo habitual es que sea verde porque es un color relativamente poco presente en la ropa, pero valdría cualquier color siempre que no esté presente en el resto de los objetos de la imagen. 

También debéis conseguir alguna imagen fija y algún archivo de audio. Todo esto, como digo, podéis grabarlo vosotros mismos con el móvil o, en caso contrario, descargarlo de Internet. Para esta práctica no es imprescindible pero pensando ya en la práctica final es recomendable que intentéis utilizar contenidos propios o que tengan una licencia de Creative Commons o similar. Algunas fuentes de las que os podéis descargar contenidos interesantes son: 

\begin{itemize}
	\item {\bf Vídeos en general:} \url{https://www.pexels.com/videos}.
	\item {\bf Otra fuente para vídeos:} \url{https://www.videvo.net}
	\item {\bf Vídeos con croma:} \url{https://pixabay.com/es/videos/search/chroma/}
	\item {\bf Efectos de vídeo:} \url{https://detonationfilms.com/Stock_Directory.html} 
	\item {\bf Efectos de sonido:} \url{https://freesound.org}.
	\item {\bf Imágenes:} \url{https://pixabay.com/es/}
	\item {\bf Música:} \url{https://www.freemusicarchive.org/search}
	\item {\bf Más música:} \url{https://incompetech.com/music/}
	\item {\bf Y más música:} \url{https://machinimasound.com}
	\item {\bf Y todavía más música:} \url{https://www.youtube.com/audiolibrary}
	\item {\bf Y todavía más música:} \url{https://www.youtube.com/c/FreeMusicAudioLibraryNow}
\end{itemize}
 


\section{Desarrollo de la práctica}

Debéis generar un vídeo con DaVinci Resolve en el que apliquéis, al menos, las siguientes técnicas:

\begin{enumerate}
	\item {\bf Montaje de secuencias: }Unir tres o cuatro secuencias de vídeo utilizando transiciones entre ellas. 
	\item {\bf Banda sonora: } Añadir una banda sonora al vídeo, bien sea de música, de voz, o ambas. 
	\item {\bf Croma: }Utilizar un vídeo grabado con un croma para insertar encima otro vídeo o una imagen. 
	\item {\bf Seguimiento de objetos: } Realizar el seguimiento de algún objeto de un vídeo utilizándolo para manejar el movimiento de otro objeto. 

\end{enumerate}


\section{Materiales a entregar}

\begin{itemize}
	\item {\bf Archivos del proyecto: }Cread un archivo zip con todos los archivos empleados (imágenes, vídeos y audio) y el archivo del proyecto DaVinci (archivo .drp).
	\item {\bf Vídeo definitivo: } Generad el vídeo definitivo de vuestro proyecto. Para ello, en la pestaña ``Deliver'', seleccionad uno de los presets, pulsad en ``Add to render queue'' y después en ``Start render'' a la derecha. Si el tamaño del archivo es demasiado grande podéis reeducir su tamaño (y la calidad), seleccionando el preset ``Custom'' con una resolución no demasiado grande (720p es más que suficiente), formato mp4 y en el parámetro ``Quality'' reducir la tasa binaria incluso hasta 1000 Kb/s.
\end{itemize}


\end{document}

	
