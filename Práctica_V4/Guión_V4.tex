\documentclass[es,practica]{uah}

\tema{V4}
\titulo{Animación 2D con Synfig Studio}{}

\begin{document}

\titulacion{Grados Informática}
\asignatura{Sistemas Audiovisuales y Aplicaciones Multimedia}{}
\curso{2021/2022} 

\maketitle




\section{Introducción}

El objetivo de esta práctica es crear una animación 2D de corta duración similar a la realizada en el vídeo que acompaña a la práctica. 

\section{Tareas a realizar}

Realizar una animación en la que se utilicen, al menos, los siguientes elementos:

\begin{itemize}
	\item Creación de un fondo, puede ser muy simple como el del vídeo, o algo más elaborado, y animación del mismo.
	\item Creación de algún tipo de personaje (muñeco, animal, etc.)
	\item Creación de un esqueleto asociado al personaje
	\item Animación del personaje utilizando el esqueleto
\end{itemize}

\section{Materiales a entregar}

Archivo con el proyecto Synfig (.sifz). Incluye también todos los archivos multimedia que hayas empleado, si es el caso.


\end{document}

	
