\documentclass[es,practica]{uah}

\tema{V5}
\titulo{Modelado de objetos con Blender}{}

\begin{document}

\titulacion{Grados Informática}
\asignatura{Sistemas Audiovisuales y Aplicaciones Multimedia}{}
\curso{2021/2022}

\maketitle


\section{Introducción}

El objetivo de la práctica es realizar un renderizado de un objeto 3D utilizando algunas de las principales funciones que ofrece Blender. 

Podéis replicar directamente la figura que se puede ver en los vídeos, o realizar cualquier otra de vuestro gusto. Los únicos requisitos son que manejéis, al menos, las herramientas que se utilizan en el vídeo: 

\begin{itemize}
	\item Edición manual de la malla de un objeto
	\item Herramientas de esculpido
	\item Manejo de la luz
	\item Materiales
	\item Renderizado final
\end{itemize}

\section{Materiales a entregar}

Debéis entregar:
\begin{itemize}
	\item Fichero \emph{.blender} con el proyecto realizado. 
	\item Imagen renderizada, en formato \emph{.png}.
\end{itemize}






\end{document}

	
